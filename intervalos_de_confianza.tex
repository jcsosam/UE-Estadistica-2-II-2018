%%%%%%%%%%%%%%%%%%%%%%%%%%%%%%%%%%%%%%%%%
% Beamer Presentation
% LaTeX Template
% Version 1.0 (10/11/12)
%
% This template has been downloaded from:
% http://www.LaTeXTemplates.com
%
% License:
% CC BY-NC-SA 3.0 (http://creativecommons.org/licenses/by-nc-sa/3.0/)
%
%%%%%%%%%%%%%%%%%%%%%%%%%%%%%%%%%%%%%%%%%

%----------------------------------------------------------------------------------------
%	PACKAGES AND THEMES
%----------------------------------------------------------------------------------------

\documentclass[usenames,dvipsnames,9pt]{beamer}

\mode<presentation> {

% The Beamer class comes with a number of default slide themes
% which change the colors and layouts of slides. Below this is a list
% of all the themes, uncomment each in turn to see what they look like.

%\usetheme{default}
%\usetheme{AnnArbor}
%\usetheme{Antibes}
%\usetheme{Bergen}
%\usetheme{Berkeley}
%\usetheme{Berlin}
%\usetheme{Boadilla}
%\usetheme{CambridgeUS}
%\usetheme{Copenhagen}
%\usetheme{Darmstadt}
%\usetheme{Dresden}
%\usetheme{Frankfurt}
%\usetheme{Goettingen}
%\usetheme{Hannover}
%\usetheme{Ilmenau}
%\usetheme{JuanLesPins}
%\usetheme{Luebeck}
\usetheme{Madrid}
%\usetheme{Malmoe}
%\usetheme{Marburg}
%\usetheme{Montpellier}
%\usetheme{PaloAlto}
%\usetheme{Pittsburgh}
%\usetheme{Rochester}
%\usetheme{Singapore}
%\usetheme{Szeged}
%\usetheme{Warsaw}

% As well as themes, the Beamer class has a number of color themes
% for any slide theme. Uncomment each of these in turn to see how it
% changes the colors of your current slide theme.

%\usecolortheme{albatross}
%\usecolortheme{beaver}
%\usecolortheme{beetle}
%\usecolortheme{crane}
%\usecolortheme{dolphin}
%\usecolortheme{dove}
%\usecolortheme{fly}
%\usecolortheme{lily}
%\usecolortheme{orchid}
%\usecolortheme{rose}
%\usecolortheme{seagull}
%\usecolortheme{seahorse}
%\usecolortheme{whale}
%\usecolortheme{wolverine}

%\setbeamertemplate{footline} % To remove the footer line in all slides uncomment this line
%\setbeamertemplate{footline}[page number] % To replace the footer line in all slides with a simple slide count uncomment this line

%\setbeamertemplate{navigation symbols}{} % To remove the navigation symbols from the bottom of all slides uncomment this line
}

\usepackage{graphicx} % Allows including images
\usepackage{booktabs} % Allows the use of \toprule, \midrule and \bottomrule in tables
\usepackage{amsfonts}
\usepackage{amsmath}
\usepackage{amssymb}
\usepackage{color, colortbl}
\usepackage{flafter}
\usepackage{enumerate}
\usepackage[T1]{fontenc}
\usepackage[latin1]{inputenc}
\usepackage{amsthm,latexsym,array}
\usepackage{float}
\usepackage{geometry}
\usepackage{mathtools}
\usepackage{multicol}
\usepackage{tcolorbox}
\usepackage{rotating}
\usepackage{ragged2e}
\usepackage{setspace}
\usepackage{subfigure}


\usefonttheme[onlymath]{serif}
\newcommand{\btVFill}{\vskip0pt plus 1filll}
%
\def\e#1{{\rm e}^{#1}}
\def\exp#1{{\rm exp}{#1}}
\def\frac#1#2{{{#1}\over{#2}}}
\def\binom#1#2{{{#1}\choose{#2}}}
\def\spot{$\bullet$\hspace{0.1cm}}
\def\le{\left}
\def\ri{\right}
\def\pro{\propto}
\def\prop{\propto}
%
\DeclareMathOperator*{\tr}{tr}
\DeclareMathOperator*{\logit}{logit}
\DeclareMathOperator*{\argmax}{arg\,max}
\DeclareMathOperator*{\argmin}{arg\,min}
%
\newcommand\iid{\mathrel{\overset{\makebox[0pt]{\mbox{\normalfont\tiny\sffamily iid}}}{\sim}}}
\newcommand\simiid{\mathrel{\overset{\makebox[0pt]{\mbox{\normalfont\tiny\sffamily iid}}}{\sim}}}
\newcommand\simind{\mathrel{\overset{\makebox[0pt]{\mbox{\normalfont\tiny\sffamily ind}}}{\sim}}}
\newcommand\eqd{\mathrel{\overset{\makebox[0pt]{\mbox{\normalfont\tiny\sffamily d}}}{=}}}
\newcommand{\ind}[1]{\mathbb{I}\left\{ #1 \right\}}
\newcommand{\pr}[1]{\mathbb{P}\text{r}\left[#1\right]}
\newcommand{\expec}[1]{\mathbb{E}\left[#1\right]}
\newcommand{\expe}[1]{\mathbb{E}\left[#1\right]}
\newcommand{\var}[1]{\mathbb{V}\text{ar}\left[#1\right]}
\newcommand{\sd}[1]{\text{SD}\left[#1\right]}
\newcommand{\cov}[1]{\mathbb{C}\text{ov}\left[#1\right]}
\newcommand{\coefvar}[1]{\text{CV}\left[#1\right]}
\newcommand{\diag}[1]{\text{diag}\left[#1\right]}
\newcommand{\expo}[1]{\exp{ \left\{ #1 \right\}}}
\newcommand{\ex}[1]{\exp{ \left\{ #1 \right\}}}
\newcommand{\quo}[1]{\textquotedblleft#1\textquotedblright}
\newcommand{\comi}[1]{\textquotedblleft#1\textquotedblright}
\newcommand\floor[1]{\lfloor#1\rfloor}
\newcommand{\ra}{\sqrt}
\newcommand{\bs}{\boldsymbol}
%
\def\teh{\hat{\theta}}\def\tevh{\hat{\boldsymbol{\theta}}}
\def\Unif{\small{\mathsf{Unif}}}
\def\MN{\small{\mathsf{Mult}}}
\def\Cat{\small{\mathsf{Cat}}}
\def\Dir{\small{\mathsf{Dir}}}
\def\DP{\small{\mathsf{DP}}}
\def\Ber{\small{\mathsf{Ber}}}
\def\Bin{\small{\mathsf{Bin}}}
\def\BetaBin{\small{\mathsf{BetaBin}}}
\def\NegBin{\small{\mathsf{NegBin}}}
\def\Nor{\small{\mathsf{N}}}
\def\normal{\small{\mathsf{N}}}
\def\Bet{\small{\mathsf{Beta}}}
\def\bet{\small{\mathsf{Beta}}}
\def\Gamd{\small{\mathsf{Gam}}}
\def\IGamd{\small{\mathsf{IGam}}}
\def\IG{\small{\mathsf{IGam}}}
\def\tdistr{\small{\mathsf{t}}}
\def\hyphen{\text{\textendash}}
%%% ----------------------------------------------------------------------
% FIGURES ----------------------------------------------------------------
%=-=-=-=-=-=-=-=-=-=-=-=-=-=-=-=-=-=-=-=-=-=-=-=-=-=-=-=-=-=-=-=-=-=-=-=-=
%\graphicspath{{C:/PROJECT/presentation/figs/}}
%=-=-=-=-=-=-=-=-=-=-=-=-=-=-=-=-=-=-=-=-=-=-=-=-=-=-=-=-=-=-=-=-=-=-=-=-=



%TITLE PAGE
\title[Intervalos]{Intervalos de Confianza} % The short title appears at the bottom of every slide, the full title is only on the title page
\subtitle{ }
%\newline \textbf{JASA}, Vol. 91, No. 433 (Mar., \textbf{1996}), pp. 142--153}

\author[Juan Sosa]{\LARGE{Juan Sosa, PhD}} % Your name
\institute[Universidad Externado] % Your institution as it will appear on the bottom of every slide, may be shorthand to save space
{
%University of California, Santa Cruz \\ % Your institution for the title page
\begin{figure}[h!]
\centering
\includegraphics[scale=.17]{./figs/logo-UE.pdf}
\end{figure}
}
\date{ } % Date, can be changed to a custom date

%\logo{\includegraphics[height=1.4cm]{Grateful_Slug.png}}


%\AtBeginSection[]
%{
%  \begin{frame}<beamer>
%    \frametitle{Outline}
%    \tableofcontents[currentsection]
%  \end{frame}
%}



\begin{document}

\begin{frame}

\titlepage % Print the title page as the first slide

\end{frame}

%%%%%%%%%%%%%%%%
%%% OVERVIEW %%%
%%%%%%%%%%%%%%%%

%\begin{frame}
%\frametitle{Overview} % Table of contents slide, comment this block out to remove it
%\tableofcontents % Throughout your presentation, if you choose to use \section{} and \subsection{} commands, these will automatically be printed on this slide as an overview of your presentation
%\end{frame}


\begin{frame}{Proporci�n poblacional}

\begin{block}{Modelo}
	$$X_1,X_2,\ldots,X_n\simiid \Ber(\pi)$$
\end{block}

\begin{block}{Intervalo de confianza (bilaterial)}
	$$
	\text{IC}_{100(1-\alpha)\%}\left( \pi \right) = \hat{\pi} \pm z_{1-\alpha/2}\, \sqrt{ \frac{\hat{\pi}(1-\hat{\pi})}{n} }
	$$
	Nota: Esta aproximaci�n es apropiada para $n\geq30$, $n\hat{\pi} \geq 5$ y $n(1-\hat{\pi}) \geq 5$.
\end{block}

\begin{block}{Tama�o de muestra}
	$$
	n = \frac{(z_{1-\alpha/2})^2\,\pi_0(1-\pi_0)}{ME^2}
	$$
	donde $\pi_0$ es la proporci�n muestral de una de un estudio piloto (por ejemplo) y $ME$ es el margen de error.
\end{block}


\end{frame}




\begin{frame}{Media poblacional}

\begin{block}{Modelo}
	$$X_1,X_2,\ldots,X_n\simiid \Nor(\mu, \sigma^2)$$
\end{block}

\begin{block}{Intervalo de confianza (bilaterial)}
		$$
		\text{IC}_{100(1-\alpha)\%}\left( \mu \right) =  \bar{x} \pm \tdistr_{n-1, 1-\alpha/2}\,\frac{s}{\sqrt{n}}
		$$
		Nota: Para $n\geq 30$ se tiene que $\tdistr\approx \Nor(0,1)$ y la poblaci�n no tiene que ser Normal.
	\end{block}

\begin{block}{Tama�o de muestra}
	$$
	n = \frac{(z_{1-\alpha/2})^2\,\sigma^2_0}{ME^2}
	$$
	donde $\sigma^2_0$ es la varianza muestral de un estudio piloto (por ejemplo) y $ME$ es el margen de error.
\end{block}


\end{frame}




\begin{frame}{Varianza poblacional poblacional}

\begin{block}{Modelo}
	$$X_1,X_2,\ldots,X_n\simiid \Nor(\mu, \sigma^2)$$
\end{block}

\begin{block}{Intervalo de confianza (bilateral)}
	$$
	\text{IC}_{100(1-\alpha)\%}\left( \sigma^2 \right) = 
	\left( \frac{(n-1)s^2}{\chi^2_{n-1,1-\alpha/2}} \,;\,  \frac{(n-1)s^2}{\chi^2_{n-1,\alpha/2}} \right)
	$$
donde $\chi^2_{n-1}$ denota la distribuci�n chi cuadrado con $n-1$ grados de libertad.
\end{block}


\end{frame}




\begin{frame}{Diferencia de proporciones poblacional}

\begin{block}{Modelo}
	Poblaciones independientes:
	\vspace{-5pt}
	$$X_1,X_2,\ldots,X_{n_1}\simiid \Ber(\pi_1)\qquad Y_1,Y_2,\ldots,Y_{n_2}\simiid \Ber(\pi_2)$$
	
\end{block}


\begin{block}{Intervalo de confianza (bilateral)}
	$$
	\text{IC}_{100(1-\alpha)\%}\left( \pi_1 - \pi_2 \right) = 
	\hat{\pi}_1 - \hat{\pi}_2 \pm  z_{1 - \alpha/2} \sqrt{ \frac{\hat{\pi}_1(1-\hat{\pi}_1)}{n_1} + \frac{\hat{\pi}_2(1-\hat{\pi}_2)}{n_2} }
	$$
	Nota: Esta aproximaci�n es apropiada para
	$$n_1,n_2\geq30 \quad n_1\hat{\pi}_1,n_1\hat{\pi}_2 \geq 5 \quad n_1(1-\hat{\pi}_1), n_2(1-\hat{\pi}_2) \geq 5$$
\end{block}

\end{frame}



\begin{frame}{Diferencia de medias poblacional}

\begin{block}{Modelo}
	Poblaciones independientes:
	\vspace{-5pt}
	$$X_1,X_2,\ldots,X_{n_1}\simiid \Nor(\mu_1,\sigma^2_1)\qquad Y_1,Y_2,\ldots,Y_{n_2}\simiid \Nor(\mu_2,\sigma^2_2)$$
	
\end{block}


\begin{block}{Intervalo de confianza (bilateral) bajo homogenedidad ($\sigma^2_1 = \sigma^2_2$)}
	$$
	\text{IC}_{100(1-\alpha)\%}\left( \mu_1-\mu_2 \right) = 
	\bar{x_1} - \bar{x}_2  \pm t_{n_1+n_2-2,1-\alpha/2}\, s_p\sqrt{\frac{1}{n_1} + \frac{1}{n_2}}
	$$
	donde
	$$
	s^2_p = \frac{(n_1-1)s^2_1 + (n_2-1)s^2_2}{n_1 + n_2 -2}
	$$
	es la varianza conjugada de las muestras.
	
	Nota: Para $n\geq 30$ se tiene que $\tdistr\approx \Nor(0,1)$ y la poblaci�n no tiene que ser Normal.
\end{block}

\end{frame}



\begin{frame}{Diferencia de medias poblacional}

\begin{block}{Modelo}
	Poblaciones independientes:
	\vspace{-5pt}
	$$X_1,X_2,\ldots,X_{n_1}\simiid \Nor(\mu_1,\sigma^2_1)\qquad Y_1,Y_2,\ldots,Y_{n_2}\simiid \Nor(\mu_2,\sigma^2_2)$$
	
\end{block}


\begin{block}{Intervalo de confianza (bilateral) bajo heterogenedidad ($\sigma^2_1 \neq \sigma^2_2$)}
	$$
		\text{IC}_{100(1-\alpha)\%}\left( \mu_1-\mu_2 \right) = 
	\bar{x_1} - \bar{x}_2  \pm t_{\nu,1-\alpha/2}\,\sqrt{\frac{s^2_1}{n_1} + \frac{s^2_2}{n_2}}
	$$
	donde
	$$
	\nu = \frac{\left(\frac{s^2_1}{n_1} + \frac{s^2_2}{n_2}\right)^2}{\frac{1}{n_1-1}\left(\frac{s^2_1}{n_1}\right)^2 + \frac{1}{n_2-1}\left(\frac{s^2_2}{n_2}\right)^2}
	$$
	corresponde a los grados de libertad.
	
	Nota: Para $n\geq 30$ se tiene que $\tdistr\approx \Nor(0,1)$ y la poblaci�n no tiene que ser Normal.
\end{block}

\end{frame}




\begin{frame}{Cociente de varianzas poblacional}

\begin{block}{Modelo}
	Poblaciones independientes:
	\vspace{-5pt}
	$$X_1,X_2,\ldots,X_{n_1}\simiid \Nor(\mu_1,\sigma^2_1)\qquad Y_1,Y_2,\ldots,Y_{n_2}\simiid \Nor(\mu_2,\sigma^2_2)$$
	
\end{block}

\begin{block}{Intervalo de confianza (bilateral)}
$$
\text{IC}_{100(1-\alpha)\%}\left(\frac{\sigma^2_1}{\sigma^2_2}\right) = \left(F_{\alpha/2,n_2-1,n_1-1}\frac{s^2_1}{s^2_2} ; F_{1-\alpha/2,n_2-1,n_1-1}\frac{s^2_1}{s^2_2} \right)
$$
donde $F_{n_2-1,n_1-1}$ denota la distribuci�n $F$ con $n_2-1$ grados de libertad en el numerador y $n_1-1$ grados de libertad en el denominador.
\end{block}


\end{frame}




\begin{frame}{Distribuci�n $F$ (Fisher-Snedecor)}

\begin{block}{ }
La funci�n de densidad de una variable aleatoria $F$ con $r_1$ grados de libertad del numerador y $r_2$ grados de libertad del denominador es:
$$
f(x) = \dfrac{(r_1/r_2)^{r_1/2}\Gamma[(r_1+r_2)/2]x^{(r_1/2)-1}}{\Gamma[r_1/2]\Gamma[r_2/2][1+(r_1x/r_2)]^{(r_1+r_2)/2}}\qquad x\geq 0
$$
donde $\Gamma(z) = \int_0^\infty t^{z-1}\ex{-t}\,\text{d}t$ es la funci�n gamma.
\end{block}

\begin{figure}[h!]
\centering
\includegraphics[scale=.13]{./figs/distr_F.pdf}
\end{figure}


\end{frame}



\end{document} 